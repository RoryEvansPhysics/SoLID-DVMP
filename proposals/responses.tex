\section{Responses to Items Identified in the 2016 Review}

Our 2016 SoLID Run-Group proposal was deemed to be of high scientific merit,
but there were a number of technical questions that were asked to be studied
before final approval can be given.  The following list is compiled from the
TAC, Theory, and SoLID review reports, reordered according to topic.  Our 
response to each item is also given.

\subsection{SoLID Acceptance Simulations}

{\it The simulations for this measurement may benefit from tracking DEMP
  events through the full SoLID GEANT4 simulation (GEMC), particularly for
  kinematics with the lowest momentum protons (300~MeV/c).}

\subsection{Resolution and Fermi Smearing}

{\it The effects of Fermi-smearing, detector resolution, ionization energy
  loss and bremsstrahlung need to be clarified.  Although they seem to all be
  included in Figs.~15 and 16, it was not clear which curves included which
  effects.}

{\it The authors may want to switch off $^3$He Fermi motion in their
  simulations and determine how large and in which kinematics they see a
  difference.  Having evidence of non-negligible nuclear effects at an early
  stage would encourage theorists to extend now their calculations from
  inclusive to exclusive measurements for a timely and correct utilization of
  the data the authors propose to take.  It would also be helpful to elaborate
  on the possible corrections in addition to Fermi motion, such as from binding
  and nucleon off-shell effects, as well as corrections beyond the impulse
  approximation from rescattering or final state interactions of the detected
  proton.}

{\it Fermi-momentum is not just a kinematic effect.  It also affects the DEMP
  amplitude.  The $^3$He momentum distribution $\rho(p)$ is plotted in Fig.~10
  (Appendix~A).  The weighted distribution $p^2\rho(p)$ peaks at
  $p_n\approx$60~MeV/c.  This means that the effective $x_B$ is smeared by
  $\approx p_n/M\approx$6\%.  The significance of this effect should be
  discussed.  Also, if the proton momentum resolution is good enough, it will
  be possible to correct for this effct, event-by-event.}

\subsection{Dialog with Theorists}

{\it There are a number of important theory issues raised by this proposal.
  These probably cannot be fully resolved before re-submission, but it will be
  important to have a clear dialog with relevant theorists (and
  experimentalists) in place...  Both Goloskokov and Kroll, and Liutti and
  Goldstein, have published estimates of $\sigma_T$, based on transversity GPDs
  and a twist-3 helicity-flip pion distribution amplitude.  One or the other of
  these theory groups should be engaged in a discussion of both the
  $|\sigma^y_L|$ and $|\sigma^y_{TT}|$ terms.}

{\it The QCD factorization theorem implies color transparency for the final
  state $\pi^-$ in this proposal.  Thus the $^3$He$(e,e'\pi^-)$ final state 
  interactions (FSI) are identical with $^3$He$(e,e'p)$, just with a more exotic
  scattering amplitude.  It is not practical to obtain full FSI calculations
  before resubmission, but a dialog should be started both with the groups
  doing FSI calculations, and the groups doing Deep Virtual calculations on
  light nuclei.  Empirically, it will be useful to determine if the FSI `peak'
  lies within the $^3$He$(e,e'\pi^-p)pp$ acceptance of this proposal.}

\subsection{Experimental Background}

{\it The committee is convinced that the SIDIS background is likely not a
  major problem.  However, an alternate approach (rather than SIDIS
  fragmentation functions) could be used.  The primary background channel under
  study is $^3$He$(e,e'\pi^-)pp$ with the two undetected protons as
  spectators.  The continuum background that can leak under the quasi-exclusive
  peak can be of the form $e+n\rightarrow e'+\pi^-+\Delta^+$ with the
  $\Delta^+$ decaying to $p+\pi^0$.}

\subsection{Projected Uncertainties}

{\it The collaboration should attempt to quantify the projected precision of
  the measured spin-dependent cross section.  Although the asymmetry may have a
  smaller error bar, the spin-dependent cross section difference has a simpler
  interpretation.  Measuring the spin-dependent cross section is also
  consistent with the opening sentence of Appendix A.}

{\it The extraction of the term $|\sigma^y_{TT}+2\epsilon\sigma^y_L|$ in
  Eqn.~8 from the other $\sin\beta$ and $\cos\beta$ terms requires good
  knowledge of the $\beta$-acceptance in each $t$-bin.  This should be shown,
  in addition to the acceptance plots of Fig.~12.}
